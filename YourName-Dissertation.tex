% !TEX TS-program = pdflatex
%%%%%%%%%%%%%%%%%%%%%%%%%%%%%%%%%%%%%%%%%%%%%%%%%%%%%%%%%%%%%%%
%
%     filename  = "YourName-Dissertation.tex",
%     version   = "1.6.5",
%     date      = "2019/07/10",
%     authors   = "Gary L. Gray,
%     copyright = "Gary L. Gray",
%     address   = "Engineering Science and Mechanics,
%                  212 Earth & Engineering Sciences Bldg.,
%                  Penn State University,
%                  University Park, PA 16802,
%                  USA",
%     telephone = "814-863-1778",
%     email     = "gray@psu.edu",
%
%%%%%%%%%%%%%%%%%%%%%%%%%%%%%%%%%%%%%%%%%%%%%%%%%%%%%%%%%%%%%%%
% Change History:
% The change history can be found in the accompanying document
% entitled "YourName-Dissertation Template Change History.md".
%%%%%%%%%%%%%%%%%%%%%%%%%%%%%%%%%%%%%%%%%%%%%%%%%%%%%%%%%%%%%%%
%
% This is a template file to help get you started using the
% psuthesis.cls for theses and dissertations at Penn State
% University. You will, of course, need to put the
% psuthesis.cls file someplace that LaTeX will find it.
%
% I have set up a directory structure that I find to be clean
% and convenient. You can readjust it to suit your tastes. In
% fact, the structure used by our students is even a little
% more involved and commands are defined to point to the
% various directories.
%
% This document has been set up to be typeset using pdflatex.
% About the only thing you will need to change if typesetting
% using latex is the \DeclareGraphicsExtensions command.
%
% The psuthesis document class uses the same options as the
% book class. In addition, it requires that you have the
% ifthen, calc, setspace, and tocloft packages.
%
% The first additional option specifies the degree type. You
% can choose from:
%	Ph.D. using class option <phd>
%	M.S. using class option <ms>
%	M.Eng. using class option <meng>
%	M.A. using class option <ma>
%	B.S. using class option <bs>
%	B.A. using class option <ba>
%	Honors from the Schreyer Honors College <schreyer>
%
% The second additional option inlinechaptertoc determines
% the formatting of the Chapter entries in the Table of
% Contents. The default sets them as two-line entries (try it).
% If you want them as one-line entries, issue the
% inlinechaptertoc option.
%
% The class option schreyer should be used for theses
% submitted to the Schreyer Honors College.
%
% The class option esc should be used by all Engineering Science
% students.
%
% The option option twoha should be used if you are earning
% interdisciplanary honors and thus have two honors advisors.
%
% The class option ``secondthesissupervisor'' should be used
% for baccalaureate honors degrees if you have a second
% Thesis Supervisor.
%
% The vita is only included with the phd option and it is
% placed at the end of the thesis. The permissions page is only
% included with the ms, meng, and ma options.
%%%%%%%%%%%%%%%%%%%%%%%%%%%%%%%%%%%%%%%%%%%%%%%%%%%%%%%%%%%%%%%
% Only one of the following lines should be used at a time.
% Doctoral students.
%\documentclass[phd,12pt]{psuthesis}
% Masters students
%\documentclass[ms,12pt]{psuthesis}
% Bachelors students in the Schreyer Honors College.
\documentclass[phd,12pt]{psuthesis}
% Bachelors students in the Schreyer Honors College & Engineering Science.
%\documentclass[bs,schreyer,esc,twoha,12pt]{psuthesis}
% Bachelors students in Engineering Science.
%\documentclass[bs,esc,12pt]{psuthesis}

\usepackage[T1]{fontenc}
\usepackage[utf8x]{inputenc}
\usepackage{lmodern}
\usepackage{textcomp}
\usepackage{microtype}

%%%%%%%%%%%%%%%%%%%%%%%%%%%
% Packages I like to use. %
%%%%%%%%%%%%%%%%%%%%%%%%%%%
\usepackage{amsmath}
\usepackage{amsfonts}
\usepackage{amssymb}
%\usepackage{amsthm}
%\usepackage{exscale}
%\usepackage[mathscr]{eucal}
%\usepackage{bm}
\usepackage{subfigure}
\usepackage{eqlist} % Makes for a nice list of symbols.
\usepackage[nosepfour,warning,np,debug,autolanguage]{numprint}
\usepackage{acro}
\usepackage{booktabs}
\usepackage[final]{graphicx}
\usepackage[dvipsnames]{color}
\DeclareGraphicsExtensions{.pdf, .jpg}

\usepackage{url}

% http://www.tex.ac.uk/cgi-bin/texfaq2html?label=citesort
\usepackage{cite}

\usepackage{titlesec}

%%%%%%%%%%%%%%%%%%%%%%%%%%%%%%%
% Use of the hyperref package %
%%%%%%%%%%%%%%%%%%%%%%%%%%%%%%%
%
% This is optional and is included only for those students
% who want to use it.
%
% To the hyperref package, uncomment the following line:
% Note that these links will be printed in \textcolor (usually black)
%\usepackage[hidelinks]{hyperref}
%
% If you would prefer colored links instead, uncomment
\usepackage{hyperref}
% \hypersetup{
	% colorlinks,
	% linkcolor={red!10!black},
	% citecolor={blue!50!black},
	% urlcolor={blue!80!black}
% }
%
% Note that you should also uncomment the following line:
%\renewcommand{\theHchapter}{\thepart.\thechapter}
%
% to work around some a problem hyperref has with the fact
% the psuthesis class has unnumbered pages after which page
% counters are reset.

% Set the baselinestretch using the setspace package.
% The LaTeX Companion claims that a \baselinestretch of
% 1.24 gives one-and-a-half line spacing, which is allowed
% by the PSU thesis office. As of October 18, 2013, the Graduate
% School states ``The text of an eTD may be single-, double- or
% one- and-a-half-spaced.'' Go nuts!
\setstretch{1.24}


%%%%%%%%%%%%%%%%%%%%%%%%%%%%%%%%%%%%
% SPECIAL SYMBOLS AND NEW COMMANDS %
%%%%%%%%%%%%%%%%%%%%%%%%%%%%%%%%%%%%
% Place user-defined commands below.

% Define the \acro command.
\makeatletter
\newif\ifFirstPar       \FirstParfalse
\def\smc{\sc}
\def\ninepoint{\small}
\DeclareRobustCommand\SMC{%
  \ifx\@currsize\normalsize\small\else
   \ifx\@currsize\small\footnotesize\else
    \ifx\@currsize\footnotesize\scriptsize\else
     \ifx\@currsize\large\normalsize\else
      \ifx\@currsize\Large\large\else
       \ifx\@currsize\LARGE\Large\else
        \ifx\@currsize\scriptsize\tiny\else
         \ifx\@currsize\tiny\tiny\else
          \ifx\@currsize\huge\LARGE\else
           \ifx\@currsize\Huge\huge\else
            \small\SMC@unknown@warning
 \fi\fi\fi\fi\fi\fi\fi\fi\fi\fi
}
\newcommand\SMC@unknown@warning{\TBWarning{\string\SMC: unrecognised
    text font size command -- using \string\small}}
\newcommand\textSMC[1]{{\SMC #1}}
\newcommand\acro[1]{\textSMC{#1}\@}

\makeatother

% command for vectors
\newcommand{\bv}[1]{\ensuremath{\vec{#1}}}
% command for unit vectors
\newcommand{\uv}[2][blank]{%
\ifthenelse{\equal{#1}{blank}}%
{\ensuremath{\hat{u}_{#2}}}%
{}%
%
\ifthenelse{\equal{#1}{.}}%
{\ensuremath{\dot{\hat{u}}_{#2}}}%
{}%
%
\ifthenelse{\equal{#1}{'}}%
{\ensuremath{\hat{u}'_{#2}}}%
{}%
}
\newcommand{\ui}{\ensuremath{\hat{\imath}}}
\newcommand{\uj}{\ensuremath{\hat{\jmath}}}
\newcommand{\uk}{\ensuremath{\hat{k}}}



%%%%%%%%%%%%%%%%%%%%%%%%%%%%%%%%%%%%%%%%%
% Renewed Float Parameters              %
% (Makes floats fit better on the page) %
%%%%%%%%%%%%%%%%%%%%%%%%%%%%%%%%%%%%%%%%%
\renewcommand{\floatpagefraction}{0.85}
\renewcommand{\topfraction}      {0.85}
\renewcommand{\bottomfraction}   {0.85}
\renewcommand{\textfraction}     {0.15}

% ----------------------------------------------------------- %

%%%%%%%%%%%%%%%%
% FRONT-MATTER %
%%%%%%%%%%%%%%%%
% Title
\title{Adversarial Reinforcement Learning Applications in Cyberphysical Systems Security}

% Author and Department
\author{Taha Eghtesad}
\program{Informatics}
\dept{College of Information Sciences and Technology}
% the degree will be conferred on this date
\degreedate{Dec 2023}
% year of your copyright
\copyrightyear{2023}

% This command is used for students submitting a thesis to the
% Schreyer Honors College and for students in Engineering Science.
% The argument of this command should contain every after the word
% ``requirements'' that appears on the title page. This provides the
% needed flexibility for all the degree types.
% \bachelorsdegreeinfo{for a doctorate research proposal \\ in Informatics}

% This is the document type. For example, this could also be:
%	Comprehensive Document
%	Thesis Proposal
% \documenttype{Thesis}
% \documenttype{Dissertation}
% \documenttype{Comprehensive Document}
\documenttype{Doctorate Research Proposal}


% This will generally be The Graduate School, though you can
% put anything in here to suit your needs.
\submittedto{The Doctoral Committee}

% This is the college to which you are submitting the
% thesis/dissertation.
\collegesubmittedto{\;}


%%%%%%%%%%%%%%%%%%
% Signatory Page %
%%%%%%%%%%%%%%%%%%
% You can have up to 7 committee members, i.e., one advisor
% and up to 6 readers.
%
% Begin by specifying the number of readers.
\numberofreaders{4}

% For baccalaureate honors degrees, enter the name of your
% honors advisor below.
% \honorsadvisor{Honors A. Laszka}
% {Assistant Professor of Informatics}
% \honorsadvisortwo{Honors P. Advisor, Jr.}
% {Professor of Engineering Science and Mechanics}

% For baccalaureate honors degrees, if you have a second
% Thesis Supervisor, enter his or her name below.
% \secondthesissupervisor{Second T. Supervisor}

% For baccalaureate honors degrees, certain departments
% (e.g., Engineering Science and Mechanics) require the
% signature of the department head. In that case, enter the
% name and title of your department head below.
% \escdepthead{Department Q. Head}
% \escdeptheadtitle{P. B. Breneman Chair and Professor 
% of Engineering Science and Mechanics
% }

% Input reader information below. The optional argument, which
% comes first, goes on the second line before the name.
\advisor[Dissertation Advisor][Chair of Committee]
		{Aron Laszka}
		{Assistant Professor of Information Sciences and Technology}

\readerone[Pennsylvania State University]
			{Jing Yang}
			{Associate Professor of Electrical Engineering and Computer Science}

\readertwo[Pennsylvania State University]
			{Hadi Hosseini}
			{Assistant Professor of Information Sciences and Technology}

% \readerthree[]
% 			{}
% 			{}

% \readerfour[]
% 			{}
% 			{}

% \readerfive[Optional Title Here]
% 			{Reader Name}
% 			{Professor of SomeThing}

% Format the Chapter headings using the titlesec package.
% You can format section headings and the like here too.
\definecolor{gray75}{gray}{0.75}
\newcommand{\hsp}{\hspace{15pt}}
\titleformat{\chapter}[display]{\fontsize{30}{30}\selectfont\bfseries\sffamily}{Chapter \thechapter\hsp\textcolor{gray75}{\raisebox{3pt}{|}}}{0pt}{}{}

\titleformat{\section}[block]{\Large\bfseries\sffamily}{\thesection}{12pt}{}{}
\titleformat{\subsection}[block]{\large\bfseries\sffamily}{\thesubsection}{12pt}{}{}


% Makes use of LaTeX's include facility. Add as many chapters
% and appendices as you like.
% \includeonly{%
% chapters,%
% appendices%
% Chapter-1/Chapter-1,%
% Chapter-2/Chapter-2,%
% Chapter-3/Chapter-3,%
% Chapter-4/Chapter-4,%
% Chapter-5/Chapter-5,%
% Appendix-A/Appendix-A,%
% Appendix-B/Appendix-B,%
% Appendix-C/Appendix-C,%
% Appendix-D/Appendix-D,%
% Appendix-E/Appendix-E%
% }

\usepackage{listings}

\usepackage{mymacros}

%%%%%%%%%%%%%%%%%
% THE BEGINNING %
%%%%%%%%%%%%%%%%%
\begin{document}
\pagestyle{fancy}
\fancyhead[L,C,R]{}
\fancyfoot[L,R]{}
\fancyfoot[C]{\thepage}
\renewcommand{\headrulewidth}{0pt}
\renewcommand{\footrulewidth}{0pt}
%%%%%%%%%%%%%%%%%%%%%%%%
% Preliminary Material %
%%%%%%%%%%%%%%%%%%%%%%%%
% This command is needed to properly set up the frontmatter.
\frontmatter

%%%%%%%%%%%%%%%%%%%%%%%%%%%%%%%%%%%%%%%%%%%%%%%%%%%%%%%%%%%%%%
% IMPORTANT
%
% The following commands allow you to include all the
% frontmatter in your thesis. If you don't need one or more of
% these items, you can comment it out. Most of these items are
% actually required by the Grad School -- see the Thesis Guide
% for details regarding what is and what is not required for
% your particular degree.
%%%%%%%%%%%%%%%%%%%%%%%%%%%%%%%%%%%%%%%%%%%%%%%%%%%%%%%%%%%%%%
% !!! DO NOT CHANGE THE SEQUENCE OF THESE ITEMS !!!
%%%%%%%%%%%%%%%%%%%%%%%%%%%%%%%%%%%%%%%%%%%%%%%%%%%%%%%%%%%%%%

% Generates the title page based on info you have provided
% above.
\psutitlepage

% Generates the committee page -- this is bound with your
% thesis. If this is an baccalaureate honors thesis, then
% comment out this line.
\psucommitteepage

% Generates the abstract. The argument should point to the
% file containing your abstract. 
\thesisabstract{SupplementaryMaterial/Abstract}

% Generates the Table of Contents
\thesistableofcontents

% Generates the List of Figures
\begin{singlespace}
\renewcommand{\listfigurename}{\sffamily\Huge List of Figures}
\setlength{\cftparskip}{\baselineskip}
\addcontentsline{toc}{chapter}{List of Figures}
%\fancypagestyle{plain}{%
%\fancyhf{} % clear all header and footer fields
%\fancyfoot[C]{\thepage}} % except the center
\listoffigures
\end{singlespace}
\clearpage

% Generates the List of Tables
\begin{singlespace}
\renewcommand{\listtablename}{\sffamily\Huge List of Tables}
\setlength{\cftparskip}{\baselineskip}
\addcontentsline{toc}{chapter}{List of Tables}
\listoftables
\end{singlespace}
\clearpage

% Generates the List of Symbols. The argument should point to
% the file containing your List of Symbols. 
\thesislistofsymbols{SupplementaryMaterial/ListOfSymbols}

% Generates the Acknowledgments. The argument should point to
% the file containing your Acknowledgments. 
\thesisacknowledgments{SupplementaryMaterial/Acknowledgments}

% Generates the Epigraph/Dedication. The first argument should
% point to the file containing your Epigraph/Dedication and
% the second argument should be the title of this page. 
%\thesisdedication{SupplementaryMaterial/Dedication}{Dedication}



%%%%%%%%%%%%%%%%%%%%%%%%%%%%%%%%%%%%%%%%%%%%%%%%%%%%%%
% This command is needed to get the main part of the %
% document going.                                    %
%%%%%%%%%%%%%%%%%%%%%%%%%%%%%%%%%%%%%%%%%%%%%%%%%%%%%%
\thesismainmatter

%%%%%%%%%%%%%%%%%%%%%%%%%%%%%%%%%%%%%%%%%%%%%%%%%%
% This is an AMS-LaTeX command to allow breaking %
% of displayed equations across pages. Note the  %
% closing the "}" just before the bibliography.  %
%%%%%%%%%%%%%%%%%%%%%%%%%%%%%%%%%%%%%%%%%%%%%%%%%%
\allowdisplaybreaks{
%\pagestyle{fancy}
%\fancyhead{}
%
%%%%%%%%%%%%%%%%%%%%%%
% THE ACTUAL CONTENT %
%%%%%%%%%%%%%%%%%%%%%%
% Chapters
\chapter{Introduction}
\label{chap:intro}

With the increasing prevalence of Advanced Persistent Threats (APTs) posed by determined adversaries to cyberphysical systems, even the most secure software is susceptible to data breaches. Therefore, defending against these threats is of paramount importance. In recent years, computational approaches have supplanted model-based defenses, as model-based approaches are limited by the understanding of how a complex CPS works. Deep Reinforcement Learning, as a sophisticated sequential decision-making paradigm, has emerged as a pivotal tool for detecting and mitigating APT attacks and devising corresponding mitigation strategies in CPS.

Markov Decision Processes (MDPs) provide a formal framework for modeling sequential decision-making in reinforcement learning. In an MDP, there are two main components: agents and environments. The agent interacts with the environment, which is characterized by a fully or partially observed state, a representation of its current situation. The agent can take actions to influence the environment, leading to a state transition that is governed by stationary transition dynamics while receiving a reward based on how well the agent performed in achieving its objective. Within this framework, agents employ a policy, a strategy that defines how they select actions based on the current observed state. The fundamental objective of reinforcement learning is to maximize the discounted future rewards over time.

In many practical applications of reinforcement learning in cyber-physical systems planning and security, the RL algorithms must grapple with MDPs featuring high-dimensional action and observation spaces. In Deep Reinforcement Learning (DRL), the strategy is modeled using a parameterized deep neural network that is optimized using gradient-based search, leading to scalability challenges. This challenge is often referred to as the 'curse of dimensionality,' where the gradient diminishes given the numerous dimensions of the search space. 

For instance, consider the task of safeguarding against false information injection in navigation applications. Here, adversaries seek to deceitfully mark roads and intersections as 'high-traffic' under the pretense of 'road work,' thereby rerouting vehicles onto longer or less desirable paths. To develop a robust defense strategy, it is imperative first to understand what constitutes an optimal attack. This demands the creation of a deep reinforcement learning framework for approximating the optimal attack sequence. Unfortunately, off-the-shelf reinforcement learning algorithms falter in this scenario, given the complexity involved in choosing which roads to falsely designate as blocked, especially in the context of a city-wide transportation graph featuring thousands of street links and hundreds of intersections.

In adversarial reinforcement learning settings, as in the case of defense against APTs, the traditional MDP framework is extended to accommodate multiple agents, each with its own policy, actively shaping the environment's dynamics. This setting often involves two primary actors: an attacker and a defender, both striving to optimize their respective objectives. The defender seeks to protect and secure a system, while the attacker attempts to compromise it. Both agents compete in real-time, leading to dynamic and constantly evolving strategies. The environment is no longer stationary and undergoes continuous transformations in response to the actions of these competing agents, invalidating the stationary assumption of the MDP.

To illustrate, consider a moving target defense scenario, where a defender strives to reconfigure assets under their control, such as servers and networked devices, in a manner that compels adversaries to remain caught in an endless cycle of exploration while probing these assets. In response, adversaries may adapt their reconnaissance strategies to evade detection. This adaptive behavior nullifies the value of reconnaissance patterns learned by the defender, necessitating the defender's ability to adapt to both old and new strategies.

In our pursuit, we aim to address these core challenges:

\begin{itemize}
    \item[Scalability] 
        \begin{enumerate}
            \item Develop a DRL framework for identifying an optimal policy for adversaries when no defenses exist in a complex CPS.
            
            \item Determine an optimal defense strategy as the best response to an adversary's attack strategy under the assumption that the attack strategy remains static and stationary.
        \end{enumerate}
    \item[Adaptability]
        \begin{enumerate}
            \item A \textbf{black-box} model, where adversaries lack access to the defender's strategy and can only infer it through interactions with the environment.
            
            \item A \textbf{white-box} model, where adversaries possess access to the defender's strategy parameter models.
        \end{enumerate}
\end{itemize}

In addressing these challenges, we aim to enhance the applicability of reinforcement learning in the ever-evolving landscape of cybersecurity and sequential decision-making.
\chapter{Literature Review}
\label{chap:literature}

\iTaha{I have copy pasted these paragraphs from older papers.}

\section{Reinforcement Learning for cybersecurity}

Application of machine learning---especially \textit{deep reinforcement learning} (DRL)---for cybersecurity has gained attention recently. Nguyen~\ea~\cite{nguyen2019deep} surveyed current literature on applications of DRL on cybersecurity. These applications include: DRL-based security methods for cyber-physical systems, autonomous intrusion detection techniques~\cite{iannucci2019performance}, and multi-agent DRL-based game-theoretic simulations for defense strategies against cyber attacks.

For example, Malialis~\cite{malialis2015distributed,malialis2015distributed2} applied multi-agent deep reinforcement learning on network routers to throttle the processing rate in order to prevent \textit{distributed denial of service} (DDoS) attacks. Bhosale~\ea~\cite{bhosale2014cooperative} proposed a cooperative multi-agent reinforcement learning for intelligent systems~\cite{herrero2009multiagent} to enable quick responses. Another example for multi-agent reinforcement learning is the fuzzy $Q$-Learning approach for detecting and preventing intrusions in \textit{wireless sensor networks} (WSN)~\cite{shamshirband2014cooperative}. Furthermore, Tong~\ea~\cite{tong2019finding} proposed a multi-agent reinforcement learning framework for alert correlation based on double oracles.

\subsection{Reinforcement Learning for Moving Target Defense}

\subsection{Reinforcement Learning for False Information Injection}

\section{Model based mitigation}

\section{Reinforcement Learning Approaches}

The basics of deep reinforcement learning are described here: D$Q$L, DDPG, DSPG, TRPO, PPO.

\subsection{Multi-Agent Reinforcement Learning}

Deep Multi-Agent RL belongs here. MADDPG, QMix, etc.

\subsection{Hierarchical RL}

Hierarchical Reinforcement Learning (HRL) has gained significant attention due to its applications and development. These methods have proven to be successful in tasks that require coordination between multiple agents, such as Unnamed Aerial Vehicles (UAVs) and autonomous vehicles, to complete objectives efficiently.

For instance, Yang~\ea~\cite{yang2018hierarchical} devised a general framework for combining compound and basic tasks in robotics, such as navigation and motor functions, respectively. However, they limited the application to single-agent RL at both levels. Similarly, Chen et al. used attention networks to incorporate environmental data with steering functions of autonomous vehicles in a hierarchical RL manner so that the vehicle can safely and smoothly change lanes.

In the UAV applications, Zhang~\ea~\cite{zhang2020hierarchical} demonstrated the success of hierarchical RL in the coordination of wireless communication and data collection of UAVs.

Although our problem is in a different domain, the fundamental ideas of these works are applicable to us since we are dealing with cooperation and coordination between adversarial agents in finding an optimal manipulation strategy in navigation applications. The study results indicate that the use of Reinforcement Learning approaches accurately modeled the effects of false information injection on navigation apps.

\subsection{Graph Convoloutional Networks}

\section{Game Theory}
\chapter{Research Proposal}
\label{chap:proposal}

\section{Problem Statement}

\section{Modeling}

\iTaha{I copy pasted this from the older paragraph}
\subsection{False Information Injection in Navigation Application}

\subsubsection{Environment}

The traffic model is defined by a \emph{road network} $G = (V, E)$, where $V$ is a set of nodes representing road intersections, and $E$ is a set of directed edges representing road segments between the intersections.
Each edge $e \in E$ is associated with a tuple $e = \langle t_e, b_e, c_e, p_e \rangle$, where $t_e$ is the free flow time of the edge, $c_e$ is the capacity of the edge, and $b_e$ and $p_e$ are the parameters for the edge to calculate actual edge travel time $W_e(n_e)$ given the congestion of the network, where $n_e$ is the number of vehicles currently traveling along the edge~\cite{transportationnetworks}.
Specifically, we use the following function for $W_e(n_e)$:
\begin{align}
    W_e(n) = t_e \times \left(1 + b_e \left(\frac{n_e}{c_e}\right)^{p_e}\right)
\end{align}

The \emph{set of vehicle trips} are given with $R$, where each trip $r \in R$ is a tuple $\langle o_r, d_r, s_r \rangle$, with $o_r \in V$ and $d_r \in V$ the origin and destination of the trip, respectively, and $s_r$ the number of vehicles traveling between the origin-destination pair $\langle o_r,d_r \rangle$. 
%We assume that each trip $r$ pertains to a number of vehicles traveling from the same source to the same destination; $s_r$ is the number of such vehicles. 

\subsubsection{State Transition}
\label{sec:state}

For each vehicle trip $r \in R$ at each time step $t \in \mathbb{N}$, \emph{vehicle location} $l_r^t \in V \cup (E \times \mathbb{N})$ represents the location of vehicle $r$ at the end of time step $t$, where the location is either a node in $V$ or a tuple consisting of an edge in $E$ and a number in $\mathbb{N}$, which represents the number of timesteps left to traverse the edge.

Each vehicle trip begins at its origin; hence $l_r^0 = o_r$. At each timestep $t \in \mathbb{N}$, for each vehicle trip $r$ that $l_r^{t-1} \in V \setminus \{ d_r \}$, i.e., the vehicle trip is at a node but has not reached its destination yet, let $\oslash^{t-1}_r = (l_r^{t-1}, e_1, v_1, e_2, v_2, \ldots, e_k, d_r)$ be a shortest path from $l_r^{t-1}$ to $d_r$ considering congested travel times $\vect{w}^t$ as edge weights.
Then $l_r^t = \langle e_1, \lfloor w_e^{t-1} \rceil \rangle$, where the travel time of edge $e$ is
\begin{align}
    w_e^t = W_e\left(\sum_{\left\{r \in R \, \middle| \, l_r^{t-1} = \langle e, \cdot \rangle \right\}} r_s \right).
\end{align}
Thus, for a trip $r$ with $l_r^{t-1} = \langle e, n \rangle$, i.e., the vehicle is traveling along an edge, if $n = 1$, that is, the vehicle is one time step from reaching the next intersection, $l_r^t = v_1$. Otherwise, $l_r^t = \langle e, n-1 \rangle$. 


\subsubsection{Attacker Model}

At the high level, our attack model involves adversarial perturbations to \emph{observed} (rather than actual) travel times along edges $e$, subject to a perturbation budget constraint $B \in \mathbb{N}$.
Let $a_e^t \in \mathbb{R}$ denote the adversarial perturbation to the observed travel time over the edge $e$.
The budget constraint is then modeled as $\|\vect{a}^t\|_1 \le B$, where $\vect{a}^t$ combines all perturbations over individual edges into a vector.
%At each time step $t$, the adversary decides on a budget $B^t \in \mathbb{N}$, which can perturb the observed travel times such that or each edge $e \in E$ and each time step $t \in \mathbb{N}$, \emph{travel-time observation perturbation} $a_e^t \in \mathbb{R}$ represents the amount by which the adversary changes the observed travel time of edge $e$ during time step $t$, subject to $|| \vect{a}^t ||_1 = B_t$. Hence, for each edge $e \in E$, the observed travel time be
The observed travel time over an edge $e$ is then
\begin{align}
    \hat{w}_e^t = w_e^t + a_e^t.
\end{align}
It is this observed travel time that is then used by the vehicles to calculate their shortest paths from their current positions in the traffic network to their respective destinations.
%In the presence of an adversary, vehicle trips calculate their shortest path to their destination with perturbed travel times $\hat{w}_e^t$ rather than congested travel times $w_e^t$.
Since we aim to develop a defense that is robust to informational assumptions about the adversary, we assume that the attacker completely observes the environment at each time step $t$, including the structure of the transit network $G$, all of the trips $R$, and the current state of each trip $l_r$. %based on which it can decide on how to perturb the edges given its budget.

The attacker's goal is to maximize the total vehicle travel times, which we formalize as the following optimization problem:
\begin{align}
\label{E:attack}
    \max_{\left\{\vect{a}^1, \vect{a}^2, \ldots\right\}: \, \forall t \left( || \vect{a}^t ||_1 = B \right)} ~ \sum_{t = 0}^\infty \gamma^t \cdot \sum_{\left\{ r \in R \, | \, l_r^t \neq d_r \right\}} r_s,
\end{align}
where $\gamma \in (0, 1)$ is a temporal discount factor.

\subsubsection{Defender Model}

The defender observes the edge travel times $\vect{\hat{w}}^t$ at each time $t$, and aims to learn a detector $D(\vect{\hat{w}}^t)$ that takes observed travel times as input, and returns a prediction whether or not these are due to an adversarial perturbation.
%The defender aims to detect attacks and raise the alarm by observing the reported edge travel times by the vehicles. The detector outputs whether the observed travel times are a result of perturbation or not.
%\begin{align}
%    D(\vect{\hat{w}}^t) \mapsto d \in \{0, 1\}
%\end{align}
If the defender identifies an ongoing attack at time $t$, all future perturbations are thereby prevented, i.e., $\vect{a}^\tau = 0$ for all $\tau \geq t$.
Failure to detect attacks entails direct consequences in terms of increased travel times as formalized in the attack model in Equation~\eqref{E:attack} (which the defender aims to minimize).
On the other hand, false positives (alerts triggered when no attack is taking place) incur a fixed cost $c$.

%By detecting an attack correctly at timestep $t$, the defender prevents further perturbations of the edge travel times
%\begin{align}
%    \forall_{\tau \geq t}: \vect{a}^\tau = 0
%\end{align}

\subsection{Moving Target Defense}

\subsection{Resilient Control}


\section{DRL Approaches}

\subsection{Hierarchical Multi-Agent Adversarial RL}

\subsection{Black-Box Model}

PSRO Algorithm description

\subsection{White-Box Model}
%%%%%%%%%%%%%%%%%%%%%%%%%%%%%%%%%%%%%%%%%%%%%%%%%%%%%%%%%%%%%%%
% Appendices
%
% Because of a quirk in LaTeX (see p. 48 of The LaTeX
% Companion, 2e), you cannot use \include along with
% \addtocontents if you want things to appear the proper
% sequence.
%%%%%%%%%%%%%%%%%%%%%%%%%%%%%%%%%%%%%%%%%%%%%%%%%%%%%%%%%%%%%%%
\appendix
\titleformat{\chapter}[display]{\fontsize{30}{30}\selectfont\bfseries\sffamily}{Appendix \thechapter\textcolor{gray75}{\raisebox{3pt}{|}}}{0pt}{}{}
% If you have a single appendix, then to prevent LaTeX from
% calling it ``Appendix A'', you should uncomment the following two
% lines that redefine the \thechapter and \thesection:
%\renewcommand\thechapter{}
%\renewcommand\thesection{\arabic{section}}
% % !TEX root = ../YourName-Dissertation.tex
\Appendix{Title of the Second Appendix}

\section{Introduction}
When in the Course of human events, it becomes necessary for one people  to dissolve the political bands which have connected them with another,  and to assume among the powers of the earth, the separate and equal station  to which the Laws of Nature and of Nature's God entitle them, a decent respect to the opinions of mankind requires that they should declare  the causes which impel them to the separation.

\section{More Declaration}

We hold these truths to be self-evident, that all men are created equal,  that they are endowed by their Creator with certain unalienable Rights,  that among these are Life, Liberty and the pursuit of Happiness. --That to secure these  rights, Governments are instituted among Men, deriving their just powers  from the consent of the governed, --That whenever any Form of Government  becomes destructive of these ends, it is the Right of the People to alter  or to abolish it, and to institute new Government, laying its foundation on  such principles and organizing its powers in such form, as to them shall  seem most likely to effect their Safety and Happiness. Prudence, indeed, will dictate that Governments long established should not  be changed for light and transient causes; and accordingly all experience  hath shewn, that mankind are more disposed to suffer, while evils are  sufferable, than to right themselves by abolishing the forms to which they  are accustomed. But when a long train of abuses and usurpations, pursuing invariably the same  Object evinces a design to reduce them under absolute Despotism, it is their  right, it is their duty, to throw off such Government, and to provide new Guards for their future security. --Such has been the patient sufferance of these Colonies; and such is now the  necessity which constrains them to alter their former Systems of Government.  The history of the present King of Great Britain [George III] is a history  of repeated injuries and usurpations, all having in direct object the  establishment of an absolute Tyranny over these States. To prove this, let Facts be submitted to a candid world.
%%%%%%%%%%%%%%%%%%%%%%%%%%%%%%%%%%%%%%%%%%%%%%%%%%%%%%%%%%%%%%%
% ESM students need to include a Nontechnical Abstract as the %
% last appendix.                                              %
%%%%%%%%%%%%%%%%%%%%%%%%%%%%%%%%%%%%%%%%%%%%%%%%%%%%%%%%%%%%%%%
% This \include command should point to the file containing
% that abstract.
%\include{nontechnical-abstract}
%%%%%%%%%%%%%%%%%%%%%%%%%%%%%%%%%%%%%%%%%%%
} % End of the \allowdisplaybreak command %
%%%%%%%%%%%%%%%%%%%%%%%%%%%%%%%%%%%%%%%%%%%

%%%%%%%%%%%%%%%%
% BIBLIOGRAPHY %
%%%%%%%%%%%%%%%%
% You can use BibTeX or other bibliography facility for your
% bibliography. LaTeX's standard stuff is shown below. If you
% bibtex, then this section should look something like:
	\begin{singlespace}
	\bibliographystyle{GLG-bibstyle}
	\addcontentsline{toc}{chapter}{Bibliography}
	\bibliography{Biblio-Database}
	\end{singlespace}

%\begin{singlespace}
%\begin{thebibliography}{99}
%\addcontentsline{toc}{chapter}{Bibliography}
%\frenchspacing

%\bibitem{Wisdom87} J. Wisdom, ``Rotational Dynamics of Irregularly Shaped Natural Satellites,'' \emph{The Astronomical Journal}, Vol.~94, No.~5, 1987  pp. 1350--1360.

%\bibitem{G&H83} J. Guckenheimer and P. Holmes, \emph{Nonlinear Oscillations, Dynamical Systems, and Bifurcations of Vector Fields}, Springer-Verlag, New York, 1983.

%\end{thebibliography}
%\end{singlespace}

\backmatter

% Vita
\vita{SupplementaryMaterial/Vita}

\end{document}

