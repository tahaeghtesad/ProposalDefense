\chapter{Research Proposal}
\label{chap:proposal}

\section{Problem Statement}

\section{Modeling}

\iTaha{I copy pasted this from the older paragraph}
\subsection{False Information Injection in Navigation Application}

\subsubsection{Environment}

The traffic model is defined by a \emph{road network} $G = (V, E)$, where $V$ is a set of nodes representing road intersections, and $E$ is a set of directed edges representing road segments between the intersections.
Each edge $e \in E$ is associated with a tuple $e = \langle t_e, b_e, c_e, p_e \rangle$, where $t_e$ is the free flow time of the edge, $c_e$ is the capacity of the edge, and $b_e$ and $p_e$ are the parameters for the edge to calculate actual edge travel time $W_e(n_e)$ given the congestion of the network, where $n_e$ is the number of vehicles currently traveling along the edge~\cite{transportationnetworks}.
Specifically, we use the following function for $W_e(n_e)$:
\begin{align}
    W_e(n) = t_e \times \left(1 + b_e \left(\frac{n_e}{c_e}\right)^{p_e}\right)
\end{align}

The \emph{set of vehicle trips} are given with $R$, where each trip $r \in R$ is a tuple $\langle o_r, d_r, s_r \rangle$, with $o_r \in V$ and $d_r \in V$ the origin and destination of the trip, respectively, and $s_r$ the number of vehicles traveling between the origin-destination pair $\langle o_r,d_r \rangle$. 
%We assume that each trip $r$ pertains to a number of vehicles traveling from the same source to the same destination; $s_r$ is the number of such vehicles. 

\subsubsection{State Transition}
\label{sec:state}

For each vehicle trip $r \in R$ at each time step $t \in \mathbb{N}$, \emph{vehicle location} $l_r^t \in V \cup (E \times \mathbb{N})$ represents the location of vehicle $r$ at the end of time step $t$, where the location is either a node in $V$ or a tuple consisting of an edge in $E$ and a number in $\mathbb{N}$, which represents the number of timesteps left to traverse the edge.

Each vehicle trip begins at its origin; hence $l_r^0 = o_r$. At each timestep $t \in \mathbb{N}$, for each vehicle trip $r$ that $l_r^{t-1} \in V \setminus \{ d_r \}$, i.e., the vehicle trip is at a node but has not reached its destination yet, let $\oslash^{t-1}_r = (l_r^{t-1}, e_1, v_1, e_2, v_2, \ldots, e_k, d_r)$ be a shortest path from $l_r^{t-1}$ to $d_r$ considering congested travel times $\vect{w}^t$ as edge weights.
Then $l_r^t = \langle e_1, \lfloor w_e^{t-1} \rceil \rangle$, where the travel time of edge $e$ is
\begin{align}
    w_e^t = W_e\left(\sum_{\left\{r \in R \, \middle| \, l_r^{t-1} = \langle e, \cdot \rangle \right\}} r_s \right).
\end{align}
Thus, for a trip $r$ with $l_r^{t-1} = \langle e, n \rangle$, i.e., the vehicle is traveling along an edge, if $n = 1$, that is, the vehicle is one time step from reaching the next intersection, $l_r^t = v_1$. Otherwise, $l_r^t = \langle e, n-1 \rangle$. 


\subsubsection{Attacker Model}

At the high level, our attack model involves adversarial perturbations to \emph{observed} (rather than actual) travel times along edges $e$, subject to a perturbation budget constraint $B \in \mathbb{N}$.
Let $a_e^t \in \mathbb{R}$ denote the adversarial perturbation to the observed travel time over the edge $e$.
The budget constraint is then modeled as $\|\vect{a}^t\|_1 \le B$, where $\vect{a}^t$ combines all perturbations over individual edges into a vector.
%At each time step $t$, the adversary decides on a budget $B^t \in \mathbb{N}$, which can perturb the observed travel times such that or each edge $e \in E$ and each time step $t \in \mathbb{N}$, \emph{travel-time observation perturbation} $a_e^t \in \mathbb{R}$ represents the amount by which the adversary changes the observed travel time of edge $e$ during time step $t$, subject to $|| \vect{a}^t ||_1 = B_t$. Hence, for each edge $e \in E$, the observed travel time be
The observed travel time over an edge $e$ is then
\begin{align}
    \hat{w}_e^t = w_e^t + a_e^t.
\end{align}
It is this observed travel time that is then used by the vehicles to calculate their shortest paths from their current positions in the traffic network to their respective destinations.
%In the presence of an adversary, vehicle trips calculate their shortest path to their destination with perturbed travel times $\hat{w}_e^t$ rather than congested travel times $w_e^t$.
Since we aim to develop a defense that is robust to informational assumptions about the adversary, we assume that the attacker completely observes the environment at each time step $t$, including the structure of the transit network $G$, all of the trips $R$, and the current state of each trip $l_r$. %based on which it can decide on how to perturb the edges given its budget.

The attacker's goal is to maximize the total vehicle travel times, which we formalize as the following optimization problem:
\begin{align}
\label{E:attack}
    \max_{\left\{\vect{a}^1, \vect{a}^2, \ldots\right\}: \, \forall t \left( || \vect{a}^t ||_1 = B \right)} ~ \sum_{t = 0}^\infty \gamma^t \cdot \sum_{\left\{ r \in R \, | \, l_r^t \neq d_r \right\}} r_s,
\end{align}
where $\gamma \in (0, 1)$ is a temporal discount factor.

\subsubsection{Defender Model}

The defender observes the edge travel times $\vect{\hat{w}}^t$ at each time $t$, and aims to learn a detector $D(\vect{\hat{w}}^t)$ that takes observed travel times as input, and returns a prediction whether or not these are due to an adversarial perturbation.
%The defender aims to detect attacks and raise the alarm by observing the reported edge travel times by the vehicles. The detector outputs whether the observed travel times are a result of perturbation or not.
%\begin{align}
%    D(\vect{\hat{w}}^t) \mapsto d \in \{0, 1\}
%\end{align}
If the defender identifies an ongoing attack at time $t$, all future perturbations are thereby prevented, i.e., $\vect{a}^\tau = 0$ for all $\tau \geq t$.
Failure to detect attacks entails direct consequences in terms of increased travel times as formalized in the attack model in Equation~\eqref{E:attack} (which the defender aims to minimize).
On the other hand, false positives (alerts triggered when no attack is taking place) incur a fixed cost $c$.

%By detecting an attack correctly at timestep $t$, the defender prevents further perturbations of the edge travel times
%\begin{align}
%    \forall_{\tau \geq t}: \vect{a}^\tau = 0
%\end{align}

\subsection{Moving Target Defense}

\subsection{Resilient Control}


\section{DRL Approaches}

\subsection{Hierarchical Multi-Agent Adversarial RL}

\subsection{Black-Box Model}

PSRO Algorithm description

\subsection{White-Box Model}